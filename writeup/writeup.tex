\documentclass{article}
\usepackage{graphicx} % Required for inserting images
\usepackage{enumitem}
\usepackage{listings}
\usepackage{xcolor}
\usepackage{fontspec}
\usepackage{amsmath}
\usepackage{hyperref}
\usepackage{float}
\usepackage{booktabs}
\usepackage{caption}
\usepackage{textgreek}

\captionsetup{labelformat=empty}

% Configure listings (no box, no line numbers)
\lstset{
  language=Python,
  basicstyle=\ttfamily\small,
  keywordstyle=\color{blue},
  commentstyle=\color{gray},
  stringstyle=\color{red},
  showstringspaces=false,
  frame=none,                % no box
  breaklines=false,          % don't allow line breaks
  keepspaces=true,
  aboveskip=1em,
  belowskip=1em
}

% Optional: to prevent page breaking manually
\newenvironment{tightcode}
  {\par\noindent\vspace{1ex}\begin{minipage}{\linewidth}}
  {\end{minipage}\vspace{1ex}}

\title{CS336 Assignment 2}
\author{Dmitry Frumkin}
\date{\today}

\begin{document}

\maketitle
\tableofcontents

\setcounter{section}{1}
\setcounter{subsection}{0}

\subsection{Profiling and Benchmarking}
\subsubsection*{Problem (benchmarking\_script)}

\begin{enumerate}[label=(\alph*)]
\item \textbf{Write a script to perform basic end-to-end benchmarking of the forward and backward passes in
your model.}

\textbf{Code:} \texttt{cs336\_systems/benchmark.py}, \texttt{cs336\_systems/benchmark.sh}, \texttt{cs336\_systems/benchmark.ipynb}

In this and subsequent problems I used an H-100 via \href{https://lambda.ai/}{lambda.ai} and hydra to run
hyperparameter sweeps.  Here, the sweeps were over (model size, context window size, number of warmup iterations).
Two sweeps were run:

\begin{itemize}
\item training: forward pass (including loss computation) and backward pass;
\item inference: forward pass in inference mode.
\end{itemize}

Every benchmark run within each sweep is a separate process.  For cleaner results, I reset the GPU in the beginning of every run.
The set-up is probably different from what Stanford students had.

The difference between inference and training forward passes is minimal, except for the much larger memory 
footprint during training, which resulted in the larger models running out of memory in training.

\item \textbf{Time the forward and backward passes for the model sizes described in §1.1.2. Use 5 warmup
steps and compute the average and standard deviation of timings over 10 measurement steps.
How long does a forward pass take? How about a backward pass? Do you see high variability
across measurements, or is the standard deviation small?}

The backward takes longer than the forward pass, and the difference is more pronounced for larger models ($2 \times$).
Computation expectedly takes longer for larger models and larger contexts (scaling linearly with the context length).
The standard deviation is generally small, though it was for some reason unusually high for the large model with the context length of 128.

\item \textbf{One caveat of benchmarking is not performing the warm-up steps. Repeat your analysis without
the warm-up steps. How does this affect your results? Why do you think this happens? Also try
to run the script with 1 or 2 warm-up steps. Why might the result still be different?}

Without warm-up, the running time as well as the standard deviation are much higher.  Particularly affected
are the first runs of training and inference sweeps (small model, context 128, warm-up 0): most likely there
is some extra initialization for the new task even though we reset the GPU every time.  In general, increasing the number
of warmup steps led to a smaller standard deviation, but not always.  Having more warmup steps helps because
there is some intentional lazy initialization and auto-tuning (different behavior for frequent vs. one-off tasks).

\end{enumerate}

\begin{table}[H]
\caption{small 128}
\begin{tabular}{rlll}
\toprule
warmup & forward infer ($\mu \pm \sigma$) & forward train ($\mu \pm \sigma$) & backward  ($\mu \pm \sigma$) \\
\midrule
0 & 102.6 ± 246.736 ms & 152.2 ± 322.945 ms & 110.2 ± 196.496 ms \\
1 & 24.4 ± 0.248 ms & 34.2 ± 3.794 ms & 43.7 ± 3.025 ms \\
2 & 24.4 ± 0.167 ms & 34.6 ± 2.814 ms & 43.0 ± 1.312 ms \\
5 & 24.0 ± 0.072 ms & 38.6 ± 3.772 ms & 44.0 ± 2.632 ms \\
\bottomrule
\end{tabular}
\end{table}

\begin{table}[H]
\caption{small 256}
\begin{tabular}{rlll}
\toprule
warmup & forward infer ($\mu \pm \sigma$) & forward train ($\mu \pm \sigma$) & backward  ($\mu \pm \sigma$) \\
\midrule
0 & 35.9 ± 37.121 ms & 49.2 ± 38.016 ms & 47.1 ± 8.117 ms \\
1 & 24.4 ± 0.155 ms & 33.7 ± 2.921 ms & 43.3 ± 3.081 ms \\
2 & 24.3 ± 0.124 ms & 33.5 ± 2.773 ms & 42.2 ± 0.897 ms \\
5 & 24.1 ± 0.207 ms & 39.6 ± 3.278 ms & 42.2 ± 0.332 ms \\
\bottomrule
\end{tabular}
\end{table}

\begin{table}[H]
\caption{small 512}
\begin{tabular}{rlll}
\toprule
warmup & forward infer ($\mu \pm \sigma$) & forward train ($\mu \pm \sigma$) & backward  ($\mu \pm \sigma$) \\
\midrule
0 & 38.6 ± 36.010 ms & 45.3 ± 39.388 ms & 57.6 ± 3.594 ms \\
1 & 27.2 ± 0.125 ms & 34.2 ± 1.387 ms & 56.3 ± 0.488 ms \\
2 & 27.3 ± 0.117 ms & 34.0 ± 3.100 ms & 56.3 ± 0.319 ms \\
5 & 27.2 ± 0.093 ms & 32.9 ± 1.104 ms & 56.2 ± 0.088 ms \\
\bottomrule
\end{tabular}
\end{table}

\begin{table}[H]
\caption{small 1024}
\begin{tabular}{rlll}
\toprule
warmup & forward infer ($\mu \pm \sigma$) & forward train ($\mu \pm \sigma$) & backward  ($\mu \pm \sigma$) \\
\midrule
0 & 73.5 ± 37.531 ms & 74.7 ± 35.548 ms & 126.7 ± 1.076 ms \\
1 & 61.9 ± 0.063 ms & 63.5 ± 0.680 ms & 127.0 ± 1.173 ms \\
2 & 61.3 ± 0.156 ms & 63.8 ± 0.930 ms & 127.4 ± 1.493 ms \\
5 & 61.7 ± 1.043 ms & 64.0 ± 0.778 ms & 127.5 ± 1.512 ms \\
\bottomrule
\end{tabular}
\end{table}

\begin{table}[H]
\caption{medium 128}
\begin{tabular}{rlll}
\toprule
warmup & forward infer ($\mu \pm \sigma$) & forward train ($\mu \pm \sigma$) & backward  ($\mu \pm \sigma$) \\
\midrule
0 & 53.4 ± 13.535 ms & 60.1 ± 18.704 ms & 68.6 ± 16.136 ms \\
1 & 48.6 ± 0.469 ms & 58.1 ± 8.031 ms & 67.2 ± 7.412 ms \\
2 & 47.6 ± 0.237 ms & 61.5 ± 1.225 ms & 64.4 ± 3.629 ms \\
5 & 47.9 ± 0.346 ms & 61.8 ± 0.729 ms & 64.8 ± 2.528 ms \\
\bottomrule
\end{tabular}
\end{table}

\begin{table}[H]
\caption{medium 256}
\begin{tabular}{rlll}
\toprule
warmup & forward infer ($\mu \pm \sigma$) & forward train ($\mu \pm \sigma$) & backward  ($\mu \pm \sigma$) \\
\midrule
0 & 53.0 ± 13.543 ms & 65.5 ± 13.513 ms & 87.5 ± 10.010 ms \\
1 & 48.4 ± 0.428 ms & 61.1 ± 1.618 ms & 84.2 ± 0.483 ms \\
2 & 48.3 ± 0.343 ms & 60.9 ± 0.888 ms & 84.2 ± 0.480 ms \\
5 & 48.1 ± 0.098 ms & 60.7 ± 0.811 ms & 84.7 ± 0.585 ms \\
\bottomrule
\end{tabular}
\end{table}

\begin{table}[H]
\caption{medium 512}
\begin{tabular}{rlll}
\toprule
warmup & forward infer ($\mu \pm \sigma$) & forward train ($\mu \pm \sigma$) & backward  ($\mu \pm \sigma$) \\
\midrule
0 & 87.5 ± 13.460 ms & 90.8 ± 12.492 ms & 170.5 ± 3.898 ms \\
1 & 82.8 ± 1.488 ms & 86.7 ± 1.144 ms & 170.7 ± 3.311 ms \\
2 & 83.6 ± 1.644 ms & 87.0 ± 1.638 ms & 171.7 ± 3.862 ms \\
5 & 85.2 ± 2.301 ms & 87.0 ± 0.925 ms & 171.2 ± 2.655 ms \\
\bottomrule
\end{tabular}
\end{table}

\begin{table}[H]
\caption{medium 1024}
\begin{tabular}{rlll}
\toprule
warmup & forward infer ($\mu \pm \sigma$) & forward train ($\mu \pm \sigma$) & backward  ($\mu \pm \sigma$) \\
\midrule
0 & 188.4 ± 6.348 ms & 194.4 ± 11.046 ms & 382.0 ± 4.829 ms \\
1 & 187.2 ± 4.308 ms & 191.4 ± 2.789 ms & 384.0 ± 4.313 ms \\
2 & 188.6 ± 4.217 ms & 193.2 ± 2.702 ms & 384.1 ± 4.491 ms \\
5 & 189.4 ± 3.466 ms & 193.2 ± 1.587 ms & 385.1 ± 2.327 ms \\
\bottomrule
\end{tabular}
\end{table}

\begin{table}[H]
\caption{large 128}
\begin{tabular}{rlll}
\toprule
warmup & forward infer ($\mu \pm \sigma$) & forward train ($\mu \pm \sigma$) & backward  ($\mu \pm \sigma$) \\
\midrule
0 & 46.8 ± 9.548 ms & 96.4 ± 15.220 ms & 103.2 ± 14.702 ms \\
1 & 44.1 ± 1.171 ms & 55.2 ± 8.828 ms & 97.3 ± 0.684 ms \\
2 & 44.3 ± 1.150 ms & 58.2 ± 12.861 ms & 97.8 ± 0.610 ms \\
5 & 44.0 ± 0.567 ms & 64.9 ± 16.638 ms & 99.4 ± 4.941 ms \\
\bottomrule
\end{tabular}
\end{table}

\begin{table}[H]
\caption{large 256}
\begin{tabular}{rlll}
\toprule
warmup & forward infer ($\mu \pm \sigma$) & forward train ($\mu \pm \sigma$) & backward  ($\mu \pm \sigma$) \\
\midrule
0 & 91.9 ± 13.390 ms & 93.5 ± 6.947 ms & 180.0 ± 4.504 ms \\
1 & 88.3 ± 2.222 ms & 90.6 ± 3.008 ms & 179.4 ± 4.045 ms \\
2 & 88.4 ± 2.566 ms & 91.8 ± 2.742 ms & 180.0 ± 3.863 ms \\
5 & 89.5 ± 2.770 ms & 91.1 ± 2.368 ms & 178.9 ± 2.677 ms \\
\bottomrule
\end{tabular}
\end{table}

\begin{table}[H]
\caption{large 512}
\begin{tabular}{rlll}
\toprule
warmup & forward infer ($\mu \pm \sigma$) & forward train ($\mu \pm \sigma$) & backward  ($\mu \pm \sigma$) \\
\midrule
0 & 169.1 ± 7.198 ms & 172.4 ± 6.147 ms & 351.3 ± 2.707 ms \\
1 & 167.2 ± 4.379 ms & 171.5 ± 3.281 ms & 352.3 ± 1.536 ms \\
2 & 168.0 ± 4.551 ms & 172.3 ± 3.487 ms & 352.1 ± 1.440 ms \\
5 & 171.0 ± 4.282 ms & 172.9 ± 3.040 ms & 353.1 ± 0.842 ms \\
\bottomrule
\end{tabular}
\end{table}

\begin{table}[H]
\caption{large 1024}
\begin{tabular}{rlll}
\toprule
warmup & forward infer ($\mu \pm \sigma$) & forward train ($\mu \pm \sigma$) & backward  ($\mu \pm \sigma$) \\
\midrule
0 & 396.0 ± 13.593 ms & 403.4 ± 10.882 ms & 815.8 ± 1.243 ms \\
1 & 393.2 ± 6.629 ms & 400.8 ± 4.225 ms & 815.4 ± 2.527 ms \\
2 & 394.9 ± 5.494 ms & 402.7 ± 3.333 ms & 818.4 ± 2.278 ms \\
5 & 395.4 ± 4.079 ms & 405.0 ± 2.002 ms & 820.8 ± 1.692 ms \\
\bottomrule
\end{tabular}
\end{table}

\begin{table}[H]
\caption{xl 128}
\begin{tabular}{rlll}
\toprule
warmup & forward infer ($\mu \pm \sigma$) & forward train ($\mu \pm \sigma$) & backward  ($\mu \pm \sigma$) \\
\midrule
0 & 82.1 ± 7.323 ms & 87.8 ± 9.464 ms & 185.9 ± 8.095 ms \\
1 & 96.8 ± 0.546 ms & 92.3 ± 11.413 ms & 183.5 ± 3.187 ms \\
2 & 79.3 ± 1.152 ms & 122.2 ± 2.762 ms & 181.2 ± 0.335 ms \\
5 & 96.9 ± 0.425 ms & 121.7 ± 0.926 ms & 182.2 ± 0.827 ms \\
\bottomrule
\end{tabular}
\end{table}

\begin{table}[H]
\caption{xl 256}
\begin{tabular}{rlll}
\toprule
warmup & forward infer ($\mu \pm \sigma$) & forward train ($\mu \pm \sigma$) & backward  ($\mu \pm \sigma$) \\
\midrule
0 & 159.7 ± 12.572 ms & 162.7 ± 12.018 ms & 334.1 ± 4.455 ms \\
1 & 155.0 ± 4.233 ms & 160.0 ± 3.298 ms & 334.3 ± 2.976 ms \\
2 & 155.9 ± 3.814 ms & 159.5 ± 2.971 ms & 334.2 ± 2.520 ms \\
5 & 158.5 ± 3.520 ms & 159.6 ± 1.943 ms & 334.9 ± 1.335 ms \\
\bottomrule
\end{tabular}
\end{table}

\begin{table}[H]
\caption{xl 512}
\begin{tabular}{rlll}
\toprule
warmup & forward infer ($\mu \pm \sigma$) & forward train ($\mu \pm \sigma$) & backward  ($\mu \pm \sigma$) \\
\midrule
0 & 330.8 ± 7.189 ms & 338.3 ± 2.728 ms & 682.3 ± 4.567 ms \\
1 & 329.8 ± 6.002 ms & 336.8 ± 2.989 ms & 684.9 ± 3.852 ms \\
2 & 330.9 ± 5.083 ms & 336.7 ± 2.443 ms & 685.8 ± 1.611 ms \\
5 & 331.1 ± 4.557 ms & 337.1 ± 0.746 ms & 685.8 ± 2.263 ms \\
\bottomrule
\end{tabular}
\end{table}

\begin{table}[H]
\caption{xl 1024}
\begin{tabular}{rlll}
\toprule
warmup & forward infer ($\mu \pm \sigma$) & forward train ($\mu \pm \sigma$) & backward  ($\mu \pm \sigma$) \\
\midrule
0 & 747.8 ± 6.460 ms & OOM & OOM \\
1 & 747.8 ± 3.741 ms & OOM & OOM \\
2 & 748.4 ± 3.014 ms & OOM & OOM \\
5 & 751.9 ± 2.029 ms & OOM & OOM \\
\bottomrule
\end{tabular}
\end{table}

\begin{table}[H]
\caption{2.7B 128}
\begin{tabular}{rlll}
\toprule
warmup & forward infer ($\mu \pm \sigma$) & forward train ($\mu \pm \sigma$) & backward  ($\mu \pm \sigma$) \\
\midrule
0 & 131.5 ± 9.775 ms & OOM & OOM \\
1 & 128.2 ± 4.355 ms & OOM & OOM \\
2 & 128.0 ± 4.658 ms & OOM & OOM \\
5 & 127.6 ± 4.384 ms & OOM & OOM \\
\bottomrule
\end{tabular}
\end{table}

\begin{table}[H]
\caption{2.7B 256}
\begin{tabular}{rlll}
\toprule
warmup & forward infer ($\mu \pm \sigma$) & forward train ($\mu \pm \sigma$) & backward  ($\mu \pm \sigma$) \\
\midrule
0 & 243.4 ± 18.224 ms & OOM & OOM \\
1 & 237.7 ± 7.375 ms & OOM & OOM \\
2 & 238.1 ± 6.333 ms & OOM & OOM \\
5 & 238.9 ± 5.518 ms & OOM & OOM \\
\bottomrule
\end{tabular}
\end{table}

\begin{table}[H]
\caption{2.7B 512}
\begin{tabular}{rlll}
\toprule
warmup & forward infer ($\mu \pm \sigma$) & forward train ($\mu \pm \sigma$) & backward  ($\mu \pm \sigma$) \\
\midrule
0 & 498.6 ± 9.691 ms & OOM & OOM \\
1 & 498.2 ± 8.167 ms & OOM & OOM \\
2 & 499.4 ± 8.183 ms & OOM & OOM \\
5 & 499.1 ± 4.962 ms & OOM & OOM \\
\bottomrule
\end{tabular}
\end{table}

\begin{table}[H]
\caption{2.7B 1024}
\begin{tabular}{rlll}
\toprule
warmup & forward infer ($\mu \pm \sigma$) & forward train ($\mu \pm \sigma$) & backward  ($\mu \pm \sigma$) \\
\midrule
0 & 1044.4 ± 11.030 ms & OOM & OOM \\
1 & 1043.2 ± 2.022 ms & OOM & OOM \\
2 & 1045.4 ± 1.675 ms & OOM & OOM \\
5 & 1047.5 ± 1.702 ms & OOM & OOM \\
\bottomrule
\end{tabular}
\end{table}


\subsubsection*{Problem (nsys\_profile)}
\subsubsection*{Problem (mixed\_precision\_accumulation)}
\begin{tightcode}
\begin{lstlisting}
import torch

s = torch.tensor(0,dtype=torch.float32)
for i in range(1000):
    s += torch.tensor(0.01,dtype=torch.float32)
print(s)

s = torch.tensor(0,dtype=torch.float16)
for i in range(1000):
    s += torch.tensor(0.01,dtype=torch.float16)
print(s)

s = torch.tensor(0,dtype=torch.float32)
for i in range(1000):
    s += torch.tensor(0.01,dtype=torch.float16)
print(s)

s = torch.tensor(0,dtype=torch.float32)
for i in range(1000):
    x = torch.tensor(0.01,dtype=torch.float16)
    s += x.type(torch.float32)
print(s)

tensor(10.0001)
tensor(9.9531, dtype=torch.float16)
tensor(10.0021)
tensor(10.0021)
\end{lstlisting}
\end{tightcode}

There are two sources of error: from 0.01 stored in binary (higher in float16) and from the result of addition
stored in binary (again higher in float16).  In the first case (everything in float32), we get minimal
error; in the second case (everything in float16) we get the highest error from both sources; in the third and
fourth cases (equivalent: implicit or explicity upcasting to float32) we get only the higher error from storing
0.01 in float16, but lower error from addition.

\subsubsection*{Problem (benchmarking\_mixed\_precision)}
\subsubsection*{Problem (memory\_profiling)}

\subsection{Optimizing Attention with FlashAttention-2}
\subsubsection*{Problem (pytorch\_attention)}

\subsection{Benchmarking JIT-Compiled Attention}
\subsubsection*{Problem (torch\_compile)}
\subsubsection*{Problem (flash\_forward)}
\subsubsection*{Problem (flash\_backward)}
\subsubsection*{Problem (flash\_benchmarking)}

\setcounter{section}{2}
\setcounter{subsection}{0}

\subsection{Single-Node Distributed Communication in PyTorch}
\subsubsection*{Problem (distributed\_communication\_single\_node)}

\subsection{A Naïve Implementation of Distributed Data Parallel Training}
\subsubsection*{Problem (naive\_ddp)}
\subsubsection*{Problem (naive\_ddp\_benchmarking)}

\subsection{Improving Upon the Minimal DDP Implementation}
\subsubsection*{Problem (minimal\_ddp\_flat\_benchmarking)}
\subsubsection*{Problem (ddp\_overlap\_individual\_parameters)}
\subsubsection*{Problem (ddp\_overlap\_individual\_parameters\_benchmarking)}
\subsubsection*{Problem (ddp\_overlap\_bucketed)}
\subsubsection*{Problem (ddp\_bucketed\_benchmarking)}

\subsection{4D Parallelism}
\subsubsection*{Problem (communication\_accounting)}

{
\renewcommand{\thesubsection}{3}
\subsection{Optimizer State Sharding}
}

\subsubsection*{Problem (optimizer\_state\_sharding)}
\subsubsection*{Problem (optimizer\_state\_sharding\_accounting)}

\end{document}